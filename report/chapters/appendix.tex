\section{Appendix}

\begin{figure}[hbt!]
  \newcounter{counter}
  \NewEnviron{example}[4]{%
    \toks0=\expandafter{\BODY}%
    \toks2={\refstepcounter{counter}\thecounter & $#1$ & $#2$ & $#3$ & $#4$}%
    \xdef\TableRowContents{\the\toks2 \the\toks0 }%
    \aftergroup\TableRowContents
  }
  \centering
  \begin{tabular}{||c | c c c | c||}
    \hline
    \# & \textbf{Patterns} & \textbf{Values} & \textbf{Partial Assignment} & \textbf{Result} \\ [0.5ex]
    \hline\hline
    \begin{example}{{?a} \land b}{c \land b}{\emptyset}{\{ {?a} \mapsto c \}}\end{example} \\ \hline
    \begin{example}{{?a} \land {?b}}{a \land b}{\{ {?a} \mapsto b \}}{\bot}\end{example} \\ \hline
    \begin{example}{{?a}}{c \land d}{\{ {?a} \mapsto {b \land c} \}}{\{ {?a} \mapsto {b \land c} \}}\end{example} \\ \hline
    \begin{example}{{?a}}{{?a} \land c}{\emptyset}{\{ {?a} \mapsto {{?a} \land b} \}}\end{example} \\ \hline
    \begin{example}{{?a} \land {?a}}{(b \lor c) \land (c \lor b)}{\emptyset}{\{ {?a} \mapsto {b \lor c} \}}\end{example} \\ \hline
    \begin{example}{{?p}(t)}{t = u}{\emptyset}{\{ {?p} \mapsto {\lambda x. x = u} \}}\end{example} \\ \hline
    \begin{example}{{?p}({?t})}{t = u}{\emptyset}{\bot}\end{example} \\ \hline
    \begin{example}{{?p}({?t})}{t = u}{\{ {?t} \mapsto t \}}{\{ {?p} \mapsto \lambda x. x = u, {?t} \mapsto t \}}\end{example} \\ \hline
    \begin{example}{{?p}({?t})}{t = u}{\{ {?p} \mapsto \lambda x. x = u \}}{\{ {?p} \mapsto \lambda x. x = u, {?t} \mapsto t \}}\end{example} \\ \hline
    \begin{example}{{?p}({?t}); {?t}}{t = u; t}{\emptyset}{\{ {?p} \mapsto \lambda x. x = u; {?t} \mapsto t \}}\end{example} \\ \hline
    \begin{example}{{?f}({?a} \land b)}{(a \land b) \lor (b \land a)}{\emptyset}{\bot}\end{example} \\ \hline
    \begin{example}{{?f}({?a} \land b); {?a}}{(a \land b) \lor (b \land a); a}{\emptyset}{\{ {?a} \mapsto a, {?f} \mapsto \lambda x. x \lor x \}}\end{example} \\ \hline
    \begin{example}{\exists x. x}{\exists y. y}{\emptyset}{\{ x \mapsto y \}}\end{example} \\ \hline
    \begin{example}{{?f}(\exists x. x)}{(\exists y. y) \land (\exists z. z)}{\emptyset}{\{ {?f} \mapsto \exists y. y, x \mapsto y \}}\end{example} \\ \hline
    \begin{example}{\exists x. {?a}}{\exists y. y = u}{\emptyset}{\bot}\end{example} \\ \hline
    \begin{example}{\exists x. {?p}(x)}{\exists y. y = u}{\emptyset}{\{ x \mapsto x, {?p} \mapsto \lambda x. x = u \}}\end{example} \\ \hline
    % \begin{example}{a}{a}{a}{a}\end{example} \\ \hline
    % \begin{example}{a}{a}{a}{a}\end{example} \\ \hline
  \end{tabular}
  \caption[Matching examples]{Selected examples to illustrate the matching procedure.
    Patterns and values are all formulas; types of the other symbols should be inferred consequently.}
  \label{fig:matching-examples}
\end{figure}

\begin{figure}[hbt!]
  \begingroup
  \allowdisplaybreaks
  \begin{align*}
  % Hypothesis
  \frac{}{{?a} \vdash {?a}} \quad (\text{H.}) && \\[1em]
  % And
  \frac{{?a}, {?b} \vdash}{{?a} \land {?b} \vdash} \quad (\text{I.L.} \land) && \frac{\vdash {?a} \quad \vdash {?b}}{\vdash {?a} \land {?b}} \quad (\text{I.R.} \land) \\[1em]
  \frac{{?a} \land {?b} \vdash}{{?a}, {?b} \vdash} \quad (\text{E.L.} \land) && \frac{\vdash {?a} \land {?b}}{\vdash {?a}} \quad (\text{E.R.} \land_1) \quad\quad \frac{\vdash {?a} \land {?b}}{\vdash {?b}} \quad (\text{E.R.} \land_2) \\[1em]
  % Or
  \frac{{?a} \vdash \quad {?b} \vdash}{{?a} \lor {?b} \vdash} \quad (\text{I.L.} \lor) && \frac{\vdash {?a}, {?b}}{\vdash {?a} \lor {?b}} \quad (\text{I.R.} \lor) \\[1em]
  \frac{{?a} \lor {?b} \vdash}{{?a} \vdash} \quad (\text{E.L.} \lor_1) \quad\quad \frac{{?a} \lor {?b} \vdash}{{?b} \vdash} \quad (\text{E.L.} \lor_2) && \frac{\vdash {?a} \lor {?b}}{\vdash {?a}, {?b}} \quad (\text{E.R.} \lor) \\[1em]
  % Implies
  \frac{\vdash {?a} \quad {?b} \vdash}{{?a} \Rightarrow {?b} \vdash} \quad (\text{I.L.} \Rightarrow) && \frac{{?a} \vdash {?b}}{\vdash {?a} \Rightarrow {?b}} \quad (\text{I.R.} \Rightarrow) \\[1em]
  \frac{{?a} \Rightarrow {?b} \vdash}{\vdash {?a}} \quad (\text{E.L.} \Rightarrow_1) \quad\quad \frac{{?a} \Rightarrow {?b} \vdash}{{?b} \vdash} \quad (\text{E.L.} \Rightarrow_2) && \frac{\vdash {?a} \Rightarrow {?b}}{{?a} \vdash {?b}} \quad (\text{E.R.} \Rightarrow) \\[1em]
  % Iff
  \frac{{?a} \Rightarrow {?b}, {?b} \Rightarrow {?a} \vdash}{{?a} \Leftrightarrow {?b} \vdash} \quad (\text{I.L.} \Leftrightarrow) && \frac{\vdash {?a} \Rightarrow {?b} \quad \vdash {?b} \Rightarrow {?a}}{\vdash {?a} \Leftrightarrow {?b}} \quad (\text{I.R.} \Leftrightarrow) \\[1em]
  \frac{{?a} \Leftrightarrow {?b} \vdash}{{?a} \Rightarrow {?b}, {?b} \Rightarrow {?a} \vdash} \quad (\text{E.L.} \Leftrightarrow) && \frac{\vdash {?a} \Leftrightarrow {?b}}{\vdash {?a} \Rightarrow {?b}} \quad (\text{E.R.} \Leftrightarrow_1) \quad\quad \frac{\vdash {?a} \Leftrightarrow {?b}}{\vdash {?b} \Rightarrow {?a}} \quad (\text{E.R.} \Leftrightarrow_2) \\[1em]
  % Not
  \frac{\vdash {?a}}{\neg {?a} \vdash} \quad (\text{I.L.} \neg) && \frac{{?a} \vdash}{\vdash \neg {?a}} \quad (\text{I.R.} \neg) \\[1em]
  \frac{\neg {?a} \vdash}{\vdash {?a}} \quad (\text{E.L.} \neg) && \frac{\vdash \neg {?a}}{{?a} \vdash} \quad (\text{E.R.} \neg) \\[1em]
  % Eq. refl.
  \frac{{?s} = {?s} \vdash}{\vdash} \quad (\text{E.L.} =) && \frac{}{\vdash {?s} = {?s}} \quad (\text{I.R.} =) \\[1em]
  \end{align*}
  \endgroup
  \caption[Rules]{Predefined rules provided in the front. For clarity, contexts have been omitted: the intended semantic is the one described in ???. ``H", ``I", ``E", ``L" and ``R" respectively stand for ``Hypothesis", ``Introduction", ``Elimination", ``Left" and ``Right". In the source, shorthand for these names are used, for instance ``I.L.$\land$" would be identified by \code{introLAnd}. Most introduction rules map to a single kernel proof step. Elimination rules on the other hand rely on the cut rule.}
  \label{fig:rules-list-1}
\end{figure}

\begin{figure}[hbt!]
  \begingroup
  \allowdisplaybreaks
  \begin{align*}
  % Forall
  \frac{{?p}({?t}) \vdash}{\forall x. {?p}(x) \vdash} \quad (\text{I.L.} \forall) && \frac{\vdash {?p}(\overline{?t})}{\vdash \forall x. {?p}(x)} \quad (\text{I.R.} \forall) \\[1em]
  \frac{\forall x. {?p}(x) \vdash}{{?p}({?t}) \vdash} \quad (\text{E.L.} \forall) && \frac{\vdash \forall x. {?p}(x)}{\vdash {?p}(\overline{?t})} \quad (\text{E.R.} \forall) \\[1em]
  % Exists
  \frac{{?p}(\overline{?t}) \vdash}{\exists x. {?p}(x) \vdash} \quad (\text{I.L.} \exists) && \frac{\vdash {?p}({?t})}{\vdash \exists x. {?p}(x)} \quad (\text{I.R.} \exists) \\[1em]
  \frac{\exists x. {?p}(x) \vdash}{{?p}(\overline{?t}) \vdash} \quad (\text{E.L.} \exists) && \frac{\vdash \exists x. {?p}(x)}{\vdash {?p}({?t})} \quad (\text{E.R.} \exists) \\[1em]
  % Iff subst.
  \frac{{?f}({?a}) \vdash \quad \vdash {?a} \Leftrightarrow {?b}}{{?f}({?b}) \vdash} \quad (\text{I.L.S.} \Leftrightarrow) && \frac{\vdash {?f}({?a}) \quad \vdash {?a} \Leftrightarrow {?b}}{\vdash {?f}({?b})} \quad (\text{I.R.S.} \Leftrightarrow) \\[1em]
  \frac{{?f}({?a}) \vdash \quad \vdash {?a} \Leftrightarrow {?b}}{{?f}({?b}) \vdash} \quad (\text{E.L.S.} \Leftrightarrow) && \frac{\vdash {?f}({?a}) \quad \vdash {?a} \Leftrightarrow {?b}}{\vdash {?f}({?b})} \quad (\text{E.R.S.} \Leftrightarrow) \\[1em]
  % Eq. subst.
  \frac{{?p}({?s}) \vdash \quad \vdash {?s} = {?t}}{{?p}({?t}) \vdash} \quad (\text{I.L.S.} =) && \frac{\vdash {?p}({?s}) \quad \vdash {?s} = {?t}}{\vdash {?p}({?t})} \quad (\text{I.R.S.} =) \\[1em]
  \frac{{?p}({?s}) \vdash \quad \vdash {?s} = {?t}}{{?p}({?t}) \vdash} \quad (\text{E.L.S.} =) && \frac{\vdash {?p}({?s}) \quad \vdash {?s} \Leftrightarrow {?t}}{\vdash {?p}({?t})} \quad (\text{E.R.S.} =) \\[1em]
  \end{align*}
  \endgroup
  \caption[Rules, continued]{Continuation of figure \ref{fig:rules-list-1}. Schemas marked with an overline are constrained to resolve to fresh schemas; namely schemas that do not appear elsewhere in the sequent, apart from this binding.}
  \label{fig:rules-list-2}
\end{figure}

\begin{figure}[hbt!]
  \centering
  \tiny
$\begin{array}{rll}
-4 & \text{Import}~                        & \vdash \forall x, z. (x \in \mathcal{U}(z)) \Leftrightarrow \exists y. (x \in y) \land (y \in z) \\
-3 & \text{Import}~                        & \vdash \forall x, y. (\forall z. (z \in x) \Leftrightarrow (z \in y)) \Leftrightarrow (x = y) \\
-2 & \text{Import}~                        & \vdash \forall x, y. (x \subseteq y) \Leftrightarrow \forall z. (z \in x) \Rightarrow (z \in y) \\
-1 & \text{Import}~                        & \vdash \forall x, y. (x \in \mathcal{P}(y)) \Leftrightarrow (x \subseteq y) \\
 0 & \text{Hypo.}~                         & \forall z. ({?s} \in \mathcal{U}(z)) \Leftrightarrow \exists y. ({?s} \in y) \land (y \in z) \vdash \forall z. ({?s} \in \mathcal{U}(z)) \Leftrightarrow \exists y. ({?s} \in y) \land (y \in z) \\
 1 & \text{Left}~{\forall}~0               & \forall x, z. (x \in \mathcal{U}(z)) \Leftrightarrow \exists y. (x \in y) \land (y \in z) \vdash \forall z. ({?s} \in \mathcal{U}(z)) \Leftrightarrow \exists y. ({?s} \in y) \land (y \in z) \\
 2 & \text{Cut}~{-4}, 1                    & \vdash \forall z. ({?s} \in \mathcal{U}(z)) \Leftrightarrow \exists y. ({?s} \in y) \land (y \in z) \\
 3 & \text{Hypo.}~                         & ({?s} \in \mathcal{U}({?t})) \Leftrightarrow \exists y. ({?s} \in y) \land (y \in {?t}) \vdash ({?s} \in \mathcal{U}({?t})) \Leftrightarrow \exists y. ({?s} \in y) \land (y \in {?t}) \\
 4 & \text{Left}~{\forall}~3               & \forall z. ({?s} \in \mathcal{U}(z)) \Leftrightarrow \exists y. ({?s} \in y) \land (y \in z) \vdash ({?s} \in \mathcal{U}({?t})) \Leftrightarrow \exists y. ({?s} \in y) \land (y \in {?t}) \\
 5 & \text{Cut}~2, 4                       & \vdash ({?s} \in \mathcal{U}({?t})) \Leftrightarrow \exists y. ({?s} \in y) \land (y \in {?t}) \\
 6 & \text{Hypo.}~                         & \forall y. ({?s} \in \mathcal{P}(y)) \Leftrightarrow ({?s} \subseteq y) \vdash \forall y. ({?s} \in \mathcal{P}(y)) \Leftrightarrow ({?s} \subseteq y) \\
 7 & \text{Left}~{\forall}~6               & \forall x, y. (x \in \mathcal{P}(y)) \Leftrightarrow (x \subseteq y) \vdash \forall y. ({?s} \in \mathcal{P}(y)) \Leftrightarrow ({?s} \subseteq y) \\
 8 & \text{Cut}~{-1}, 7                    & \vdash \forall y. ({?s} \in \mathcal{P}(y)) \Leftrightarrow ({?s} \subseteq y) \\
 9 & \text{Hypo.}~                         & ({?s} \in \mathcal{P}({?t})) \Leftrightarrow ({?s} \subseteq {?t}) \vdash ({?s} \in \mathcal{P}({?t})) \Leftrightarrow ({?s} \subseteq {?t}) \\
10 & \text{Left}~{\forall}~9               & \forall y. ({?s} \in \mathcal{P}(y)) \Leftrightarrow ({?s} \subseteq y) \vdash ({?s} \in \mathcal{P}({?t})) \Leftrightarrow ({?s} \subseteq {?t}) \\
11 & \text{Cut}~8, 10                      & \vdash ({?s} \in \mathcal{P}({?t})) \Leftrightarrow ({?s} \subseteq {?t}) \\
12 & \text{Hypo.}~                         & \forall y. ({?s} \subseteq y) \Leftrightarrow \forall z. (z \in {?s}) \Rightarrow (z \in y) \vdash \forall y. ({?s} \subseteq y) \Leftrightarrow \forall z. (z \in {?s}) \Rightarrow (z \in y) \\
13 & \text{Left}~{\forall}~12              & \forall x, y. (x \subseteq y) \Leftrightarrow \forall z. (z \in x) \Rightarrow (z \in y) \vdash \forall y. ({?s} \subseteq y) \Leftrightarrow \forall z. (z \in {?s}) \Rightarrow (z \in y) \\
14 & \text{Cut}~{-2}, 13                   & \vdash \forall y. ({?s} \subseteq y) \Leftrightarrow \forall z. (z \in {?s}) \Rightarrow (z \in y) \\
15 & \text{Hypo.}~                         & ({?s} \subseteq {?t}) \Leftrightarrow \forall z. (z \in {?s}) \Rightarrow (z \in {?t}) \vdash ({?s} \subseteq {?t}) \Leftrightarrow \forall z. (z \in {?s}) \Rightarrow (z \in {?t}) \\
16 & \text{Left}~{\forall}~15              & \forall y. ({?s} \subseteq y) \Leftrightarrow \forall z. (z \in {?s}) \Rightarrow (z \in y) \vdash ({?s} \subseteq {?t}) \Leftrightarrow \forall z. (z \in {?s}) \Rightarrow (z \in {?t}) \\
17 & \text{Cut}~14, 16                     & \vdash ({?s} \subseteq {?t}) \Leftrightarrow \forall z. (z \in {?s}) \Rightarrow (z \in {?t}) \\
18 & \text{Hypo.}~                         & {?v} \in {?s} \vdash {?v} \in {?s} \\
19 & \text{Right}~{\Rightarrow}~18         & \vdash ({?v} \in {?s}) \Rightarrow ({?v} \in {?s}) \\
20 & \text{Rewrite}~19                     & \setminus z. \vdash (z \in {?s}) \Rightarrow (z \in {?s}) \\
21 & \text{Right}~{\forall}~20             & \vdash \forall z. (z \in {?s}) \Rightarrow (z \in {?s}) \\
22 & \text{Rewrite}~21                     & ({?t} \in {?s}) \Rightarrow ({?t} \in {?s}) \vdash \forall z. (z \in {?s}) \Rightarrow (z \in {?s}) \\
23 & \text{Right}~{\Rightarrow}~22         & \vdash (({?t} \in {?s}) \Rightarrow ({?t} \in {?s})) \Rightarrow \forall z. (z \in {?s}) \Rightarrow (z \in {?s}) \\
24 & \text{Hypo.}~                         & ({?t} \in {?s}) \Rightarrow ({?t} \in {?s}) \vdash ({?t} \in {?s}) \Rightarrow ({?t} \in {?s}) \\
25 & \text{Left}~{\forall}~24              & \forall z. (z \in {?s}) \Rightarrow (z \in {?s}) \vdash ({?t} \in {?s}) \Rightarrow ({?t} \in {?s}) \\
26 & \text{Right}~{\Rightarrow}~25         & \vdash (\forall z. (z \in {?s}) \Rightarrow (z \in {?s})) \Rightarrow ({?t} \in {?s}) \Rightarrow ({?t} \in {?s}) \\
27 & \text{Right}~{\Leftrightarrow}~26, 23 & \vdash (\forall z. (z \in {?s}) \Rightarrow (z \in {?s})) \Leftrightarrow ({?t} \in {?s}) \Rightarrow ({?t} \in {?s}) \\
28 & \text{?Fun Instantiation}~17          & \vdash ({?s} \subseteq {?s}) \Leftrightarrow \forall z. (z \in {?s}) \Rightarrow (z \in {?s}) \\
29 & \text{Rewrite}~28                     & \vdash (\forall z. (z \in {?s}) \Rightarrow (z \in {?s})) \Leftrightarrow ({?s} \subseteq {?s}) \\
30 & \text{R. SubstIff}~27                 & (\forall z. (z \in {?s}) \Rightarrow (z \in {?s})) \Leftrightarrow ({?s} \subseteq {?s}) \vdash ({?s} \subseteq {?s}) \Leftrightarrow ({?t} \in {?s}) \Rightarrow ({?t} \in {?s}) \\
31 & \text{Cut}~29, 30                     & \vdash ({?s} \subseteq {?s}) \Leftrightarrow ({?t} \in {?s}) \Rightarrow ({?t} \in {?s}) \\
32 & \text{Hypo.}~                         & \forall y. (\forall z. (z \in {?s}) \Leftrightarrow (z \in y)) \Leftrightarrow ({?s} = y) \vdash \forall y. (\forall z. (z \in {?s}) \Leftrightarrow (z \in y)) \Leftrightarrow ({?s} = y) \\
33 & \text{Left}~{\forall}~32              & \forall x, y. (\forall z. (z \in x) \Leftrightarrow (z \in y)) \Leftrightarrow (x = y) \vdash \forall y. (\forall z. (z \in {?s}) \Leftrightarrow (z \in y)) \Leftrightarrow ({?s} = y) \\
34 & \text{Cut}~{-3}, 33                   & \vdash \forall y. (\forall z. (z \in {?s}) \Leftrightarrow (z \in y)) \Leftrightarrow ({?s} = y) \\
35 & \text{Hypo.}~                         & (\forall z. (z \in {?s}) \Leftrightarrow (z \in {?t})) \Leftrightarrow ({?s} = {?t}) \vdash (\forall z. (z \in {?s}) \Leftrightarrow (z \in {?t})) \Leftrightarrow ({?s} = {?t}) \\
36 & \text{Left}~{\forall}~35              & \forall y. (\forall z. (z \in {?s}) \Leftrightarrow (z \in y)) \Leftrightarrow ({?s} = y) \vdash (\forall z. (z \in {?s}) \Leftrightarrow (z \in {?t})) \Leftrightarrow ({?s} = {?t}) \\
37 & \text{Cut}~34, 36                     & \vdash (\forall z. (z \in {?s}) \Leftrightarrow (z \in {?t})) \Leftrightarrow ({?s} = {?t}) \\
38 & \text{?Fun Instantiation}~37          & \vdash (\forall z. (z \in {?s}) \Leftrightarrow (z \in \mathcal{U}(\mathcal{P}({?s})))) \Leftrightarrow ({?s} = \mathcal{U}(\mathcal{P}({?s}))) \\
39 & \text{Rewrite}~38                     & \vdash (\forall z. (z \in \mathcal{U}(\mathcal{P}({?s}))) \Leftrightarrow (z \in {?s})) \Leftrightarrow (\mathcal{U}(\mathcal{P}({?s})) = {?s}) \\
40 & \text{Hypo.}~                         & {?t} \in {?s} \vdash {?t} \in {?s} \\
41 & \text{Right}~{\Rightarrow}~40         & {?t} \in {?s} \vdash ({?t} \in {?s}) \Rightarrow ({?t} \in {?s}) \\
42 & \text{Weakening}~31                   & {?t} \in {?s} \vdash ({?t} \in {?s}) \Rightarrow ({?t} \in {?s}) \Leftrightarrow ({?s} \subseteq {?s}) \\
43 & \text{R. SubstIff}~41                 & ({?t} \in {?s}) \Rightarrow ({?t} \in {?s}) \Leftrightarrow ({?s} \subseteq {?s}); {?t} \in {?s} \vdash {?s} \subseteq {?s} \\
44 & \text{Cut}~42, 43                     & {?t} \in {?s} \vdash {?s} \subseteq {?s} \\
45 & \text{?Fun Instantiation}~11          & \vdash ({?s} \in \mathcal{P}({?s})) \Leftrightarrow ({?s} \subseteq {?s}) \\
46 & \text{Weakening}~45                   & {?t} \in {?s} \vdash ({?s} \subseteq {?s}) \Leftrightarrow ({?s} \in \mathcal{P}({?s})) \\
47 & \text{R. SubstIff}~44                 & ({?s} \subseteq {?s}) \Leftrightarrow ({?s} \in \mathcal{P}({?s})); {?t} \in {?s} \vdash {?s} \in \mathcal{P}({?s}) \\
48 & \text{Cut}~46, 47                     & {?t} \in {?s} \vdash {?s} \in \mathcal{P}({?s}) \\
49 & \text{Right}~{\land}~40, 48           & {?t} \in {?s} \vdash ({?t} \in {?s}) \land ({?s} \in \mathcal{P}({?s})) \\
50 & \text{Right}~{\exists}~49             & {?t} \in {?s} \vdash \exists y. ({?t} \in y) \land (y \in \mathcal{P}({?s})) \\
51 & \text{Right}~{\Rightarrow}~50         & \vdash ({?t} \in {?s}) \Rightarrow \exists y. ({?t} \in y) \land (y \in \mathcal{P}({?s})) \\
52 & \text{Hypo.}~                         & {?t} \in {?s}; {?t} \in {?u} \vdash {?t} \in {?s} \\
53 & \text{Hypo.}~                         & {?t} \in {?u} \vdash {?t} \in {?s}; {?t} \in {?u} \\
54 & \text{Left}~{\Rightarrow}~53, 52      & ({?t} \in {?u}) \Rightarrow ({?t} \in {?s}); {?t} \in {?u} \vdash {?t} \in {?s} \\
55 & \text{Left}~{\forall}~54              & \forall z. (z \in {?u}) \Rightarrow (z \in {?s}); {?t} \in {?u} \vdash {?t} \in {?s} \\
56 & \text{?Fun Instantiation}~17          & \vdash ({?u} \subseteq {?s}) \Leftrightarrow \forall z. (z \in {?u}) \Rightarrow (z \in {?s}) \\
57 & \text{Weakening}~56                   & {?t} \in {?u} \vdash (\forall z. (z \in {?u}) \Rightarrow (z \in {?s})) \Leftrightarrow ({?u} \subseteq {?s}); {?t} \in {?s} \\
58 & \text{L. SubstIff}~55                 & (\forall z. (z \in {?u}) \Rightarrow (z \in {?s})) \Leftrightarrow ({?u} \subseteq {?s}); {?t} \in {?u}; {?u} \subseteq {?s} \vdash {?t} \in {?s} \\
59 & \text{Cut}~57, 58                     & {?t} \in {?u}; {?u} \subseteq {?s} \vdash {?t} \in {?s} \\
60 & \text{?Fun Instantiation}~11          & \vdash ({?u} \in \mathcal{P}({?s})) \Leftrightarrow ({?u} \subseteq {?s}) \\
61 & \text{Weakening}~60                   & {?t} \in {?u} \vdash ({?u} \subseteq {?s}) \Leftrightarrow ({?u} \in \mathcal{P}({?s})); {?t} \in {?s} \\
62 & \text{L. SubstIff}~59                 & ({?u} \subseteq {?s}) \Leftrightarrow ({?u} \in \mathcal{P}({?s})); {?t} \in {?u}; {?u} \in \mathcal{P}({?s}) \vdash {?t} \in {?s} \\
63 & \text{Cut}~61, 62                     & {?t} \in {?u}; {?u} \in \mathcal{P}({?s}) \vdash {?t} \in {?s} \\
64 & \text{Left}~{\land}~63                & ({?t} \in {?u}) \land ({?u} \in \mathcal{P}({?s})) \vdash {?t} \in {?s} \\
65 & \text{?Fun Instantiation}~64          & \setminus y. ({?t} \in y) \land (y \in \mathcal{P}({?s})) \vdash {?t} \in {?s} \\
66 & \text{Left}~{\exists}~65              & \exists y. ({?t} \in y) \land (y \in \mathcal{P}({?s})) \vdash {?t} \in {?s} \\
67 & \text{Right}~{\Rightarrow}~66         & \vdash (\exists y. ({?t} \in y) \land (y \in \mathcal{P}({?s}))) \Rightarrow ({?t} \in {?s}) \\
68 & \text{Right}~{\Leftrightarrow}~67, 51 & \vdash (\exists y. ({?t} \in y) \land (y \in \mathcal{P}({?s}))) \Leftrightarrow ({?t} \in {?s}) \\
69 & \text{?Fun Instantiation}~5           & \vdash ({?t} \in \mathcal{U}(\mathcal{P}({?s}))) \Leftrightarrow \exists y. ({?t} \in y) \land (y \in \mathcal{P}({?s})) \\
70 & \text{R. SubstIff}~69                 & (\exists y. ({?t} \in y) \land (y \in \mathcal{P}({?s}))) \Leftrightarrow ({?t} \in {?s}) \vdash ({?t} \in \mathcal{U}(\mathcal{P}({?s}))) \Leftrightarrow ({?t} \in {?s}) \\
71 & \text{Cut}~68, 70                     & \vdash ({?t} \in \mathcal{U}(\mathcal{P}({?s}))) \Leftrightarrow ({?t} \in {?s}) \\
72 & \text{?Fun Instantiation}~71          & \setminus z. \vdash (z \in \mathcal{U}(\mathcal{P}({?s}))) \Leftrightarrow (z \in {?s}) \\
73 & \text{Right}~{\forall}~72             & \vdash \forall z. (z \in \mathcal{U}(\mathcal{P}({?s}))) \Leftrightarrow (z \in {?s}) \\
74 & \text{R. SubstIff}~73                 & (\forall z. (z \in \mathcal{U}(\mathcal{P}({?s}))) \Leftrightarrow (z \in {?s})) \Leftrightarrow (\mathcal{U}(\mathcal{P}({?s})) = {?s}) \vdash \mathcal{U}(\mathcal{P}({?s})) = {?s} \\
75 & \text{Cut}~39, 74                     & \vdash \mathcal{U}(\mathcal{P}({?s})) = {?s}
\end{array}$
\normalsize
  \caption[Sample proof]{Example of a generated kernel proof following its description in the front. This proof shows that $\vdash \mathcal{U}(\mathcal{P}({?s})) = {?s}$ with respect to the axiom of powerset, the definition of the subset relation and the axioms of extensionality and union (imports 1-4). Such a kernel proof is objectively difficult to be produced by hand; in contrast it is relatively straightforward to be achieved in the front. Note that the latex code used in this figure is also part of the output.}
  \label{fig:lisa-proof-sample}
\end{figure}

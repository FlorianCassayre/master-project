\section{The framework}
\label{sec:framework}

The work of this thesis is designated as ``a front-end for LISA''. The term ``front-end'' is generally used for layered architectures to outline a layer that is the closest to the end user. Though, what is understood as a user can vary depending on the context. In our case, we understand front-end as a programmatic interface for the user of LISA. More precisely, we were interested in designing a framework that depended on the LISA kernel and strove to provide a higher-level interface to construct and organize proofs. The use of an indefinite article in our designation is to emphasize the fact that such a layer is by essence guided by opinionated design choices (independent of LISA).

One could wonder about the necessity of relying on an existing system. The fundamental reason is to keep the core of the system sufficiently small and simple enough that it could be understood, reasoned about and perhaps formally proved by another system. We refer to this part as the \textbf{kernel} of LISA. The constraint over the complexity of the kernel gives us better reasons to trust it regarding its soundness. On the other side, the front-end does not make such claims and can thus be made arbitrarily more sophisticated. The only requirement is for the front-end to be able to reconstruct all of its proofs in the kernel, such that one could check them with the same level of confidence as if they were initially written there. We define soundness as the property for a proof in the front to have the same meaning as one in the kernel. By the requirement, soundness thus relies on the only assumption that statements in the front are logically equivalent to statements in the kernel. As a consequence such a system would satisfy the (informal) De Bruijn criterion, in the sense that the underlying proofs can be checked by a reasonably simple procedure (the kernel).

Another aspect of the front is that despite its dependency on the kernel, it does not expose it to the user. Instead, it redefines most of its components and provides mappings for uni or bi-directional translations. The motivation is two-fold. First of all, it allows us to extend the existing functionality. Practically, for example, it was decided that the kernel would not provide schematic connectors as they were not strictly needed. However the front requires such a functionality for its rules and it is not possible to emulate them at low cost, consequently we implement this functionality in the front. Alternatively we can also strengthen some of its features, for instance by representing sequents as indexed sequences rather than sets. Additionally we can redefine the method \code{toString} on all of our classes. Secondly, it allows us to enrich the DSL and bring type safety guarantees. Some types in the kernel are (purposely) weak and require runtime checks to ensure safety. Although this was not a main aspect of study in this project, it is certainly desirable and contributes to building a strong framework. This is compatible with the mid-term objective of having LISA exist as a Scala library.

Finally, the interaction between the front and the kernel is relatively limited as the front handles its own state independently. The common ground between the two are justified statements (axioms, theorems, definitions), which means that such objects can be converted back and forth. Whenever a theorem is proven in the front, it is translated to the kernel and checked for consistency. The front obtains the mirror of that justified statement and can be quite certain about its correctness.

\subsection{Language and tools}

The implementation was done in Scala version 3\footnote{Also known as Dotty: \href{https://dotty.epfl.ch}{dotty.epfl.ch}.}. The choice of version was a requirement, following the decision to upgrade LISA from version 2 to 3: because this project does not work directly on the LISA codebase, but instead depends on it as a library, both versions must be the same. Regarding the version, it turns out that some of the features implemented in this project would not have had an equivalent encoding in the older version, which lets us to confidently argue that Scala version 3 is an appropriate language for the design of DSL\footnote{DSL: domain specific language} libraries.

During the implementation, eight bugs were discovered in different areas of the Dotty compiler. All the issues that could be minimized and reproduced were reported\footnote{
\href{https://github.com/lampepfl/dotty}{github.com/lampepfl/dotty}:
\href{https://github.com/lampepfl/dotty/issues/14667}{\#14667},\
\href{https://github.com/lampepfl/dotty/issues/14707}{\#14707},\
\href{https://github.com/lampepfl/dotty/issues/14765}{\#14765},\
\href{https://github.com/lampepfl/dotty/issues/14818}{\#14818},\
\href{https://github.com/lampepfl/dotty/issues/14834}{\#14834},\
\href{https://github.com/lampepfl/dotty/issues/14858}{\#14858},\
\href{https://github.com/lampepfl/dotty/issues/14907}{\#14907}, and\
\href{https://github.com/lampepfl/dotty/issues/15145}{\#15145}.
}\textsuperscript{,}\footnote{
\href{https://github.com/scalameta/scalameta}{github.com/scalameta/scalameta}:
\href{https://github.com/scalameta/scalameta/issues/2741}{\#2741}.
}; most of them have since been fixed or are still in the process of getting fixed.

\subsection{Structure}

\begin{figure}[hbt!]
  \centering
  % https://tex.stackexchange.com/q/515582
  \begin{tikzpicture}[grow via three points={one child at (1.0,-0.6) and two children at (1.0,-0.6) and (1.0,-1.2)}, edge from parent path={(\tikzparentnode.south)|-(\tikzchildnode.west)}]
  \node{(root package)}
  child{node{\code{example}}}
  child{node{\code{front}}
      child{node{\code{fol}}}
      child{node{\code{parser}}}
      child{node{\code{printer}}}
      child{node{\code{proof}}}
      child{node{\code{theory}}}}
  child [missing] {}
  child [missing] {}
  child [missing] {}
  child [missing] {}
  child [missing] {}
  child{node{\code{util}}}
  child{node{\code{legacy}}};
  \node at (6, -0.6) {\parbox{7cm}{\hspace*{0pt}\hfill(examples)}};
  \node at (6, -1.2) {\parbox{7cm}{\hspace*{0pt}\hfill(the front-end framework)}};
  \node at (6, -1.8) {\parbox{7cm}{\hspace*{0pt}\hfill(first-order logic)}};
  \node at (6, -2.4) {\parbox{7cm}{\hspace*{0pt}\hfill(FOL parsers)}};
  \node at (6, -3.0) {\parbox{7cm}{\hspace*{0pt}\hfill(FOL and proof printers)}};
  \node at (6, -3.6) {\parbox{7cm}{\hspace*{0pt}\hfill(logic for proofs, depends on FOL)}};
  \node at (6, -4.2) {\parbox{7cm}{\hspace*{0pt}\hfill(available theories)}};
  \node at (6, -4.8) {\parbox{7cm}{\hspace*{0pt}\hfill(utilities that only rely on the kernel)}};
  \node at (6, -5.4) {\parbox{7cm}{\hspace*{0pt}\hfill(older experiments)}};
  \end{tikzpicture}
  \caption[Source code packages structure]{Structure of the packages in the source code. The reader is invited to start with the package \code{example} which presents a collection of examples demonstrating the capabilities of the framework.}
  \label{fig:packages}
\end{figure}

The code is organized hierarchically to reduce coupling. This is achieved through Scala dependent members: the components are implemented as children of traits, these traits can then extend and mix with other traits to benefit from their features (included the inherited ones). Finally, all the traits are aggregated in a singleton object that acts as a module. All the components can then be put into scope using a single import. Note that if a same member is exported from two distinct modules it will be considered as a different entity: this is because the types are path-dependent.

\section{LISA}

\begin{frame}{Proof assistants}
\framesubtitle{Formal proof verification}

Automatically verifies the validity of a proof

\vspace{0.5cm}

\centering
\begin{tikzpicture}[
block/.style={rectangle, draw, very thick, fill=gray!5, minimum width=60, minimum height=30},
none/.style={draw=none},
]
\node[none] (input) {Formal proof};
\node[block] (verifier) [right = of input] {Verifier};
\node[none] (output-true) [above right = of verifier] {$\top$};
\node[none] (output-false) [below right = of verifier] {$\bot$};

\draw[->] (input) -- (verifier);
\draw[->] (verifier.north) |- (output-true);
\draw[->] (verifier.south) |- (output-false);
\end{tikzpicture}

\end{frame}

\note{
\begin{itemize}
\item Before anything else, I will clarify the notion of proof assistants. The main objective in the area of automated reasoning is to formalize the notion of a proof such that it can be verified mechanically by a computerized program.
\item The core piece
\end{itemize}
}

\begin{frame}{Proof assistants}
\framesubtitle{Formal proof generation}

The assistant helps in producing such a proof

\vspace{0.75cm}

\centering
\scalebox{0.8}{\begin{tikzpicture}[
block/.style={rectangle, draw, very thick, fill=gray!5, minimum width=60, minimum height=30},
none/.style={draw=none},
]
\node[none] (user) {User interaction};
\node[block] (assistant) [right = of user] {Assistant};
\node[none] (input) [right = of assistant] {Formal proof};
\node[block] (verifier) [right = of input] {Verifier};
\node[none] (output-true) [above right = of verifier] {$\top$};
\node[none] (output-false) [below right = of verifier] {\color{gray}$\bot$};

\draw[->] (user) -- (assistant);
\draw[->] (assistant.north) -- ++(0cm, 1.25cm) -| (user.north);
\draw[->] (assistant) -- (input);
\draw[->] (input) -- (verifier);
\draw[->] (verifier.north) |- (output-true);
\draw[->, densely dotted] (verifier.south) |- (output-false);
\end{tikzpicture}}

\end{frame}

\note{
\begin{itemize}
\item Proof assistants use proof verifiers, however they are also able to provide feedback.
\item These formal proofs are not always straightforward to write. The proof assistant comes into play to address this limitation.
\end{itemize}
}

\begin{frame}{LISA}

\begin{itemize}
\item LISA is a proof assistant designed at LARA\footnote{\href{http://github.com/epfl-lara/lisa}{github.com/epfl-lara/lisa}} (2021 $\rightarrow$ now)
\item First-order logic and sequent calculus
\item Set theory (ZFC) as an axiomatic foundation
\item Scala 2 and 3
\end{itemize}

\end{frame}

\note{
\begin{itemize}
\item LISA is a proof assistant designed in this laboratory, which development started last year.
\item It stands out from other existing software by using different foundations: in particular it relies on an adaptation of Gentzen's sequent calculus over first-order logic, and uses axioms of set theory. It is also written in Scala and is meant to be used within Scala.
\end{itemize}
}

\begin{frame}{LISA}
\framesubtitle{Sequents and inference rules}

\centering
$$
\phi_1, ..., \phi_n \vdash \psi_1, ..., \psi_m \ \text{can be interpreted as} \left(\bigwedge_{i=1}^n \phi_i \right) \Rightarrow \left(\bigvee_{i=1}^m \psi_i \right)
$$

\vspace{0.75cm}

Examples of inference rules:

$$
\begin{prooftree}
\hypo{\vphantom{\Gamma}}
\infer1[Hypothesis]{\Gamma, \phi \vdash \phi, \Delta}
\end{prooftree}
\qquad\qquad
\begin{prooftree}
\hypo{\Gamma, \phi \vdash \Delta}
\hypo{\Sigma, \psi \vdash \Pi}
\infer2[Left $\vee$]{\Gamma, \Sigma, \phi \vee \psi \vdash \Delta, \Pi}
\end{prooftree}
$$
$$
\begin{prooftree}
\hypo{\Gamma \vdash \phi [x \mapsto t], \Delta}
\infer1[Right $\exists$]{\Gamma \vdash \exists x. \phi, \Delta}
\end{prooftree}
\qquad\qquad
\begin{prooftree}
\hypo{\Gamma \vdash \Delta}
\infer1[Inst. Pred.]{(\Gamma \vdash \Delta) [{?p} \mapsto \phi(\Psi)]}
\end{prooftree}
$$

\vspace{0.75cm} % Otherwise not centered

\end{frame}

\note{
\begin{itemize}
\item In LISA statements are expressed as sequents. A sequent is basically characterized by a pair of sets of formulas, which have the following interpretation in regular logic.
\item Then, there is also a pre-defined set of inference rules which states how one can conclude logical tautologies.
\item For instance, the hypothesis rule allows you to always conclude a sequent containing the same formula on either side.
\item Or, to obtain a disjunction as an hypothesis you must prove two other propositions.
\end{itemize}
}

\begin{frame}{LISA}
\framesubtitle{Proof}

\begin{itemize}
\item
Proof tree:\\[0.4cm]
{\centering \footnotesize
\begin{prooftree}
\hypo{}
\infer1[$\text{Hypo.}$]{{?a} \vdash {?a}; {?b}}
\infer1[$\text{Right}~{\neg}$]{\vdash \neg{?a}; {?a}; {?b}}
\hypo{}
\infer1[$\text{Hypo.}$]{{?b} \vdash \neg{?a}; {?b}}
\infer2[$\text{Left}~{\Rightarrow}$]{{?a} \Rightarrow {?b} \vdash \neg{?a}; {?b}}
\infer1[$\text{Right}~{\lor}$]{{?a} \Rightarrow {?b} \vdash \neg{?a} \lor {?b}}
\infer1[$\text{Right}~{\Rightarrow}$]{\vdash ({?a} \Rightarrow {?b}) \Rightarrow \neg{?a} \lor {?b}}
\end{prooftree}
\normalsize
 \par}
\vspace{0.5cm}
\item
Proof represented in LISA:\\[0.4cm]
{\centering \footnotesize
$\begin{array}{rlll}
0 & \text{Hypo.}~                  & {?b} \vdash \neg{?a}; {?b}                                    & [{?b}] \\
1 & \text{Hypo.}~                  & {?a} \vdash {?a}; {?b}                                        & [{?a}] \\
2 & \text{Right}~{\neg}~1          & \vdash \neg{?a}; {?a}; {?b}                                   & [{?a}] \\
3 & \text{Left}~{\Rightarrow}~2, 0 & {?a} \Rightarrow {?b} \vdash \neg{?a}; {?b}                   & [{?a}; {?b}] \\
4 & \text{Right}~{\lor}~3          & {?a} \Rightarrow {?b} \vdash \neg{?a} \lor {?b}               & [\neg{?a}; {?b}] \\
5 & \text{Right}~{\Rightarrow}~4   & \vdash ({?a} \Rightarrow {?b}) \Rightarrow \neg{?a} \lor {?b} & [{?a} \Rightarrow {?b}; \neg{?a} \lor {?b}]
\end{array}$
\normalsize
 \par}
\end{itemize}

\end{frame}

\note{Everything you want}

\begin{frame}{LISA}
\framesubtitle{Beyond proof trees}

\begin{itemize}
\item Proofs can be converted into theorems
\item Further proofs can use axioms and existing theorems as premises
\end{itemize}

\end{frame}

\note{Everything you want bis}

\begin{frame}[fragile]
\frametitle{LISA}

\begin{lstlisting}
package net.debasishg.snippet.cake

import java.util.{Date, Calendar}
object cake {

  sealed trait Currency
  case object USD extends Currency
  case object EUR extends Currency
  case object AUD extends Currency

  val baseCurrency: Currency = USD
  val test_abc = 0
}

\end{lstlisting}

\end{frame}

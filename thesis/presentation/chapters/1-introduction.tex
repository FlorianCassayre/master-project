\section{Introduction}

\makesection{
\begin{itemize}
\item .
\end{itemize}
}

\subsection{Proof assistants}

\begin{frame}{Proof assistants}
\framesubtitle{Formal proof verification}

Automatically verifies the validity of a proof

\vspace{0.5cm}

\centering
\begin{tikzpicture}[
block/.style={rectangle, draw, very thick, fill=gray!5, minimum width=60, minimum height=30},
none/.style={draw=none},
]
\node[none] (input) {Formal proof};
\node[block] (verifier) [right = of input] {Checker};
\node[none] (output-true) [above right = of verifier] {$\top$};
\node[none] (output-false) [below right = of verifier] {$\bot$};

\draw[->] (input) -- (verifier);
\draw[->] (verifier.north) |- (output-true);
\draw[->] (verifier.south) |- (output-false);
\end{tikzpicture}

\note{
\begin{itemize}
\item Before anything else, I will clarify the notion of proof assistants. One of the tasks in the area of automated reasoning is to formalize the notion of a proof such that it can be verified mechanically by a computerized program.
\item A formal proof is therefore a precise model of a mathematical proof that one would write on paper.
\item Then, as shown on this diagram we have this core element called a proof checker that takes as an input a formal proof and returns a binary result. Either the proof is accepted, which should imply that the proof is valid. Or it is rejected.
\item It should be noted that proof checkers are not meant to be complex pieces of software. This leads me to the next slide.
\end{itemize}
}

\end{frame}

\begin{frame}{Proof assistants}
\framesubtitle{Formal proof generation}

The assistant helps in producing such a proof

\vspace{0.75cm}

\centering
\scalebox{0.8}{\begin{tikzpicture}[
block/.style={rectangle, draw, very thick, fill=gray!5, minimum width=60, minimum height=30},
none/.style={draw=none},
]
\node[none] (user) {User interaction};
\node[block] (assistant) [right = of user] {Assistant};
\node[none] (input) [right = of assistant] {Formal proof};
\node[block] (verifier) [right = of input] {Checker};
\node[none] (output-true) [above right = of verifier] {$\top$};
\node[none] (output-false) [below right = of verifier] {\color{gray}$\bot$};

\draw[->] (user) -- (assistant);
\draw[->] (assistant.north) -- ++(0cm, 1.25cm) -| (user.north);
\draw (assistant) -- (input);
\draw[->] (input) -- (verifier);
\draw[->] (verifier.north) |- (output-true);
\draw[->, densely dotted] (verifier.south) |- (output-false);
\end{tikzpicture}}

\note{
\begin{itemize}
\item Proof assistants are conceptually more sophisticated than a bare proof checker. Nevertheless they rely on the same principle that ultimately a proof is considered valid if it is accepted by the proof checker. However the user is not required to write these low-level proofs themselves: the assistant generates these for them. The assistant instead provides a higher-level interface that is often compatible with interactivity. That way the user can write their proofs gradually.
\item In this project I focused on the left side of this diagram, while the right part was already provided by LISA which I am going to present now.
\end{itemize}
}

\end{frame}

\subsection{LISA}

\begin{frame}{LISA}

\begin{itemize}
\item LISA is a proof system designed at LARA\footnote{\href{http://github.com/epfl-lara/lisa}{github.com/epfl-lara/lisa}} (2021 $\rightarrow$ now)
\item First-order logic and sequent calculus
\item Set theory (ZFC) as an axiomatic foundation
\item Scala 2 and 3
\end{itemize}

\note{
\begin{itemize}
\item LISA is a proof assistant designed in this laboratory, which development started last year.
\item It stands out from other existing software by using different foundations: in particular it relies on an adaptation of Gentzen's sequent calculus over first-order logic, and uses axioms of set theory. It is also written in Scala and is meant to be used within Scala.
\end{itemize}
}

\end{frame}

\begin{frame}{LISA}
\framesubtitle{Sequents and inference rules}

\centering
$$
\phi_1, ..., \phi_n \vdash \psi_1, ..., \psi_m \ \text{can be interpreted as} \left(\bigwedge_{i=1}^n \phi_i \right) \Rightarrow \left(\bigvee_{i=1}^m \psi_i \right)
$$

\vspace{0.75cm}

Examples of inference rules:

$$
\begin{prooftree}
\hypo{\vphantom{\Gamma}}
\infer1[Hypothesis]{\Gamma, \phi \vdash \phi, \Delta}
\end{prooftree}
\qquad\qquad
\begin{prooftree}
\hypo{\Gamma, \phi \vdash \Delta}
\hypo{\Sigma, \psi \vdash \Pi}
\infer2[Left $\lor$]{\Gamma, \Sigma, \phi \lor \psi \vdash \Delta, \Pi}
\end{prooftree}
$$
$$
\begin{prooftree}
\hypo{\Gamma \vdash \phi [x \mapsto t], \Delta}
\infer1[Right $\exists$]{\Gamma \vdash \exists x. \phi, \Delta}
\end{prooftree}
\qquad\qquad
\begin{prooftree}
\hypo{\Gamma \vdash \Delta}
\infer1[Inst. Pred.]{(\Gamma \vdash \Delta) [{?p} \mapsto \phi(\Psi)]}
\end{prooftree}
$$

\vspace{0.75cm} % Otherwise not vertically centered

\note{
\begin{itemize}
\scriptsize
\item In LISA statements are expressed as sequents. A sequent is basically characterized by a pair of sets of formulas, which can be interpreted as an implication in standard logic.
\item Then, there is also a pre-defined set of inference rules which states how one can conclude logical tautologies.
\item For instance, the hypothesis rule allows you to always conclude a sequent containing the same formula on either side.
\item Or, to obtain a disjunction as an hypothesis you must prove two other propositions.
\item Rules can also express substitutions of terms, for instance with the introduction of the existence quantifier.
\item The question marks is a notation to express universally quantified variables which we call schemas. This rule basically says that we can always instantiate a predicate schema, so long as we do it across all formulas.
\end{itemize}
}

\end{frame}

\begin{frame}{LISA}
\framesubtitle{How proofs are represented}

\begin{itemize}
\item
Proof tree:\\[0.2cm]
{\centering \footnotesize
\begin{prooftree}
\hypo{}
\infer1[$\text{Hypo.}$]{{?a} \vdash {?a}; {?b}}
\infer1[$\text{Right}~{\neg}$]{\vdash \neg{?a}; {?a}; {?b}}
\hypo{}
\infer1[$\text{Hypo.}$]{{?b} \vdash \neg{?a}; {?b}}
\infer2[$\text{Left}~{\Rightarrow}$]{{?a} \Rightarrow {?b} \vdash \neg{?a}; {?b}}
\infer1[$\text{Right}~{\lor}$]{{?a} \Rightarrow {?b} \vdash \neg{?a} \lor {?b}}
\infer1[$\text{Right}~{\Rightarrow}$]{\vdash ({?a} \Rightarrow {?b}) \Rightarrow \neg{?a} \lor {?b}}
\end{prooftree}
\normalsize
 \par}
\vspace{0.5cm}
\item
Proof represented in LISA:\\[0.2cm]
{\centering \footnotesize
$\begin{array}{rlll}
0 & \text{Hypo.}~                  & {?b} \vdash \neg{?a}; {?b}                                    & [{?b}] \\
1 & \text{Hypo.}~                  & {?a} \vdash {?a}; {?b}                                        & [{?a}] \\
2 & \text{Right}~{\neg}~1          & \vdash \neg{?a}; {?a}; {?b}                                   & [{?a}] \\
3 & \text{Left}~{\Rightarrow}~2, 0 & {?a} \Rightarrow {?b} \vdash \neg{?a}; {?b}                   & [{?a}; {?b}] \\
4 & \text{Right}~{\lor}~3          & {?a} \Rightarrow {?b} \vdash \neg{?a} \lor {?b}               & [\neg{?a}; {?b}] \\
5 & \text{Right}~{\Rightarrow}~4   & \vdash ({?a} \Rightarrow {?b}) \Rightarrow \neg{?a} \lor {?b} & [{?a} \Rightarrow {?b}; \neg{?a} \lor {?b}]
\end{array}$
\normalsize
 \par}
\end{itemize}

\note{
\begin{itemize}
\item The rules presented earlier can then be used to build an entire proof for some proposition. We obtain a tree where each branch is a rule and the bottom sequent is the conclusion of the proof.
\item The above figure depicts a simple proof for the proposition appearing at the bottom.
\item In LISA the proof is represented as a directed acyclic graph, making it possible to reference the same branch multiple times (as opposed to a tree). It can be shown that such a proof has a linear representation, such as the one below.
\end{itemize}
}

\end{frame}

\begin{frame}{LISA}
\framesubtitle{Beyond proof trees}

\begin{itemize}
\item Proofs can be converted into theorems
\item Further proofs can use axioms and existing theorems as premises
\end{itemize}

\vspace{0.5cm}

\centering
\scalebox{0.75}{\begin{tikzpicture}[
block/.style={draw, rectangle, very thick, fill=gray!5, minimum width=60, minimum height=30, node distance = 0.5cm and 1.5cm},
theorem/.style={block, rounded corners},
axiom/.style={block},
]
\node[axiom] (axiom1) {Axiom 1};
\node[theorem] (theorem1) [below right = of axiom1] {Theorem 1};
\node[axiom] (axiom2) [below left = of theorem1] {Axiom 2};
\node[theorem] (theorem2) [below right = of theorem1] {Theorem 2};
\node[theorem] (theorem3) [above right = of theorem2] {Theorem 3};

\draw[<-] (axiom1) to [out=350,in=140] (theorem1);
\draw[<-] (axiom2) to [out=10,in=220] (theorem1);
\draw[<-] (theorem1) to [out=360,in=140] (theorem2);
\draw[<-] (axiom2) -- (theorem2);
\draw[<-] (theorem2) to [out=360,in=220] (theorem3);
\draw[<-] (axiom1) to [out=360,in=160] (theorem3);
\end{tikzpicture}}

\note{
\begin{itemize}
\item Then separate proofs can depend on other propositions, this is strictly controlled by a model of axioms and theorems. A theorem is a proposition with a proof, while axioms do not have an associated proof.
\item The advantage is that we only need to check every proof once, rather than every time they are used as assumptions.
\end{itemize}
}

\end{frame}

\subsection{Observations}

\begin{frame}{Observations}

\begin{itemize}
\item Inference rules are low-level and tedious to apply
\item Proof trees can only be built in one direction
\item Limited interactivity
\end{itemize}

\vspace{0.5cm}

Motivates the need of having a front-end framework

\note{
\begin{itemize}
\item As we have seen, inference rules are low-level: a simple tautology required a few steps already. They are also tedious to apply because each parameter must specified explicitly.
\item Also, proof trees can only be built forward. It is not immediately possible to start a proof from the bottom and build our way upwards.
\item The goal of this project was to improve on these aspects. For that it was decided to interface LISA, that is to design a framework having its own dialect but capable of producing proofs in LISA.
\end{itemize}
}

\end{frame}

% https://tex.stackexchange.com/a/70818
% Assumes that the two abstracts fit in one page
\newenvironment{abstractpage}
  {\cleardoublepage\vspace*{\fill}\thispagestyle{plain}}
  {\vfill\cleardoublepage}
\newenvironment{i18nabstract}[1]
  {\bigskip\selectlanguage{#1}%
   \begin{center}\bfseries\abstractname\end{center}}
  {\par\bigskip}

\begin{abstractpage}
\begin{i18nabstract}{english}
  % Don't forget to update the translation if you make any changes here!
  We propose a front-end framework for the novel proof assistant LISA. The two systems are based on Gentzen's sequent calculus with first-order logic and are implemented in the Scala programming language. Our framework supports different proof styles and provides a declarative language for tactics. The usage of tactics is facilitated thanks to a parameter inference system. The proofs written in our framework can be translated into the trusted kernel for verification. We demonstrate that our framework simplifies the process of writing formal proofs for LISA, and that it is suitable for an interactive usage. We also introduce other relevant components for LISA such as a strongly typed programming interface, a two-way parser and printer, and a proof simplifier.
\end{i18nabstract}

\begin{i18nabstract}{french}
  Nous proposons un cadriciel pour le nouvel assistant de preuve LISA. Les deux systèmes sont basés sur le calcul des séquents de Gentzen en utilisant la logique du premier ordre, et sont implémentés dans le langage de programmation Scala. Notre système prend en charge différents styles de preuve et apporte un langage déclaratif pour les tactiques. L'utilisation des tactiques est facilitée par un système d'inférence des paramètres. Les preuves écrites dans ce cadre peuvent ensuite être traduites vers le noyau de confiance pour y être vérifiées. Nous montrons que notre cadre permet de simplifier le processus d'écriture de démonstrations formelles pour LISA, et qu'il est adapté à une utilisation interactive. Nous présentons d'autres composantes clés pour LISA, telles qu'une interface de programmation fortement typée, un analyseur syntactique ains qu'un système d'impression élégante, et un simplificateur de preuves.
\end{i18nabstract}
\end{abstractpage}

\section{Conclusion}
\label{sec:conclusion}

In this project we explored the implementation of a front-end for LISA. We demonstrated the feasibility and practicality of mixing backward and forward proof styles. We introduced a concept of tactics, and we focused on the implementation of rules which provide a scalable interface to represent compound proof steps. We implemented a general matching procedure that can be used to infer most of the parameters when applying rules. We showed that our procedure was applicable to other use cases, such as the instantiation of justifications. On top of that, we also significantly improved the tooling which LISA can directly benefit from: a type safe DSL, a two-way parser/printer and various utilities to work with kernel proofs. Overall, our work was focused on interactivity and usability, and to that extent we have succeeded in producing a framework that was usable in a REPL environment. We exposed the limitations we have encountered or accidentally introduced, and provided ideas to solve them.

I would like to personally thank Viktor and Simon for giving me the opportunity to work on the early version of LISA, and later for the supervision of this thesis. I would also like to thank all of the members of the LARA laboratory for the rich discussions and precious advice I received from them.

% Crow's foot notation on tikz: https://tex.stackexchange.com/a/393568
\tikzset{
  zig zag to/.style={
    to path={(\tikztostart) -| ($(\tikztostart)!#1!(\tikztotarget)$) |- (\tikztotarget) \tikztonodes}
  },
  zig zag to/.default=0.5pt,
  one to many/.style={
    one-many, zig zag to,
  },
  oone to many/.style={
    oone-many, zig zag to,
  },
  oone to nmany/.style={
    oone-nmany, zig zag to,
  },
  one to nmany/.style={
    one-nmany, zig zag to,
  },
}
\pgfarrowsdeclare{many}{many}
{
  \pgfarrowsleftextend{+-.5\pgflinewidth}%
  \pgfarrowsrightextend{+.5\pgflinewidth}%
}
{
  \let\pgfutil@tempdima=0.6pt%
  \pgfsetdash{}{+0pt}%
  \pgfsetmiterjoin%
  \pgfpathmoveto{\pgfqpoint{0pt}{-6pt\pgfutil@tempdima}}%
  \pgfpathlineto{\pgfqpoint{-8pt\pgfutil@tempdima}{0pt}}%
  \pgfpathlineto{\pgfqpoint{0pt}{6pt\pgfutil@tempdima}}%
  \pgfpathmoveto{\pgfqpoint{0pt\pgfutil@tempdima}{0pt\pgfutil@tempdima}}%
  \pgfpathmoveto{\pgfqpoint{-8pt}{-6pt}}% 
  \pgfpathlineto{\pgfqpoint{-8pt}{-6pt}}%  
  \pgfpathlineto{\pgfqpoint{-8pt}{6pt}}% 
  \pgfusepathqstroke%
}
\pgfarrowsdeclare{nmany}{nmany}
{
  \pgfarrowsleftextend{+-.5\pgflinewidth}%
  \pgfarrowsrightextend{+.5\pgflinewidth}%
}
{
  \let\pgfutil@tempdima=0.6pt%
  \pgfsetdash{}{+0pt}%
  \pgfsetmiterjoin%
  \pgfpathmoveto{\pgfqpoint{0pt}{-6pt\pgfutil@tempdima}}%
  \pgfpathlineto{\pgfqpoint{-8pt\pgfutil@tempdima}{0pt}}%
  \pgfpathlineto{\pgfqpoint{0pt}{6pt\pgfutil@tempdima}}%
  \pgfpathmoveto{\pgfqpoint{0pt\pgfutil@tempdima}{0pt\pgfutil@tempdima}}%
  \pgfpathmoveto{\pgfqpoint{-8pt}{-6pt}}% 
  \pgfpathlineto{\pgfqpoint{-8pt}{-6pt}}%  
  %\pgfpathlineto{\pgfqpoint{-8pt}{6pt}}% 
  \pgfusepathqstroke%
}
\pgfarrowsdeclare{one}{one}
{
  \pgfarrowsleftextend{+-.5\pgflinewidth}%
  \pgfarrowsrightextend{+.5\pgflinewidth}%
}
{
  \let\pgfutil@tempdima=0.6pt%
  \pgfsetdash{}{+0pt}%
  \pgfsetmiterjoin%
  \pgfpathmoveto{\pgfqpoint{0pt\pgfutil@tempdima}{0pt\pgfutil@tempdima}}%
  \pgfpathmoveto{\pgfqpoint{-6pt}{-6pt}}% 
  \pgfpathlineto{\pgfqpoint{-6pt}{-6pt}}%  
  \pgfpathlineto{\pgfqpoint{-6pt}{6pt}}% 
  \pgfpathmoveto{\pgfqpoint{0pt\pgfutil@tempdima}{0pt\pgfutil@tempdima}}%
  \pgfpathmoveto{\pgfqpoint{-8pt}{-6pt}}% 
  \pgfpathlineto{\pgfqpoint{-8pt}{-6pt}}%  
  \pgfpathlineto{\pgfqpoint{-8pt}{6pt}}%    
  \pgfusepathqstroke%
}
\pgfarrowsdeclare{oone}{oone}
{
  \pgfarrowsleftextend{+-.5\pgflinewidth}%
  \pgfarrowsrightextend{+.5\pgflinewidth}%
}
{
  \let\pgfutil@tempdima=0.6pt%
  \pgfsetdash{}{+0pt}%
  \pgfsetmiterjoin%
  \pgfpathmoveto{\pgfqpoint{0pt\pgfutil@tempdima}{0pt\pgfutil@tempdima}}%
  \pgfpathmoveto{\pgfqpoint{-4pt}{-6pt}}%  
  \pgfpathlineto{\pgfqpoint{-4pt}{6pt}}% 
  \pgfpathmoveto{\pgfqpoint{0pt\pgfutil@tempdima}{0pt\pgfutil@tempdima}}%
  \pgfpathmoveto{\pgfqpoint{-8pt}{-6pt}}%
  \pgfpathcircle{\pgfpoint{-10pt}{0}} {3.5pt}
  \pgfusepathqstroke%
}
\begin{figure}[H]
  \centering
  \begin{tikzpicture}[auto, on grid, node distance=5cm, block/.style = {draw, fill=white, rectangle, minimum height=1cm, minimum width=2cm}, none/.style = {draw=none}]
  \node [block] (sequent) {\code{Sequent}};
  \node [block, right = of sequent] (formula) {\code{Formula}};
  \node [block, right = of formula] (term) {\code{Term}};
  \node [none] at (2.5, 0.25) {\footnotesize{is made of}};
  \node [none] at (7.5, 0.25) {\footnotesize{can combine}};
  \node [none] at (5, 1.5) {\footnotesize{can combine other}};
  \node [none] at (10, 1.5) {\footnotesize{combines other}};
  \draw [one to nmany] (sequent) -- (formula);
  \draw [oone to nmany] (formula) -- (term);
  \draw [oone to many] (formula.north) ++(-0.6, 0) |- ++(0, 0.75) |- ++(1.2, 0) -| ++(0, -0.75);
  \draw [one to nmany] (term.north) ++(-0.6, 0) |- ++(0, 0.75) |- ++(1.2, 0) -| ++(0, -0.75);
  \end{tikzpicture}
  \caption[First-order logic relationships]{First-order logic relationships presented in \textit{crow's foot notation}. Each relation corresponds to at least one syntactic combinator.}
  \label{fig:fol-erd}
\end{figure}
